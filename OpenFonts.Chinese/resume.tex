%%%%%%%%%%%%%%%%%%%%%%%%%%%%%%%%%%%%%%%
% Deedy - One Page Two Column Resume
% LaTeX Template
% Version 1.2 (16/9/2014)
%
% Original author:
% Debarghya Das (http://debarghyadas.com)
%
% Original repository:
% https://github.com/deedydas/Deedy-Resume
%
% IMPORTANT: THIS TEMPLATE NEEDS TO BE COMPILED WITH XeLaTeX
%
% This template uses several fonts not included with Windows/Linux by
% default. If you get compilation errors saying a font is missing, find the line
% on which the font is used and either change it to a font included with your
% operating system or comment the line out to use the default font.
% 
%%%%%%%%%%%%%%%%%%%%%%%%%%%%%%%%%%%%%%
% 
% TODO:
% 1. Integrate biber/bibtex for article citation under publications.
% 2. Figure out a smoother way for the document to flow onto the next page.
% 3. Add styling information for a "Projects/Hacks" section.
% 4. Add location/address information
% 5. Merge OpenFont and MacFonts as a single sty with options.
% 
%%%%%%%%%%%%%%%%%%%%%%%%%%%%%%%%%%%%%%
%
% CHANGELOG:
% v1.1:
% 1. Fixed several compilation bugs with \renewcommand
% 2. Got Open-source fonts (Windows/Linux support)
% 3. Added Last Updated
% 4. Move Title styling into .sty
% 5. Commented .sty file.
%
%%%%%%%%%%%%%%%%%%%%%%%%%%%%%%%%%%%%%%%
%
% Known Issues:
% 1. Overflows onto second page if any column's contents are more than the
% vertical limit
% 2. Hacky space on the first bullet point on the second column.
%
%%%%%%%%%%%%%%%%%%%%%%%%%%%%%%%%%%%%%%


\documentclass[]{deedy-resume-openfont}
\usepackage{fancyhdr}
    
\pagestyle{fancy}
\fancyhf{}
    
\begin{document}

%%%%%%%%%%%%%%%%%%%%%%%%%%%%%%%%%%%%%%
%
%     LAST UPDATED DATE
%
%%%%%%%%%%%%%%%%%%%%%%%%%%%%%%%%%%%%%%
\lastupdated

%%%%%%%%%%%%%%%%%%%%%%%%%%%%%%%%%%%%%%
%
%     TITLE NAME
%
%%%%%%%%%%%%%%%%%%%%%%%%%%%%%%%%%%%%%%
\namesection{张}{君 涵}{ \urlstyle{same}\href{mailto:2907402968@qq.com}{2907402968@qq.com} | 1520 6110 955
}

%%%%%%%%%%%%%%%%%%%%%%%%%%%%%%%%%%%%%%
%
%     COLUMN ONE
%
%%%%%%%%%%%%%%%%%%%%%%%%%%%%%%%%%%%%%%

\begin{minipage}[t]{0.25\textwidth} 

%%%%%%%%%%%%%%%%%%%%%%%%%%%%%%%%%%%%%%
%     EDUCATION
%%%%%%%%%%%%%%%%%%%%%%%%%%%%%%%%%%%%%%

\section{教育经历} 
\sectionsep

\subsection{电子科技大学}
\descript{硕士学位,主修电子科学与技术}
\location{2023.09-2026.06}
\sectionsep

\subsection{河海大学}
\descript{学士学位,主修电子科学与技术}
\location{2019.09-2023.06}
\sectionsep

%%%%%%%%%%%%%%%%%%%%%%%%%%%%%%%%%%%%%%
%     LINKS
%%%%%%%%%%%%%%%%%%%%%%%%%%%%%%%%%%%%%%

% \section{链接}
% \sectionsep
% Blog://  \href{http://gaocegege.com/Blog}{\bf gaocegege.com/Blog} \\
% (总计 3 万访客,8 万阅读量) \\    
% Github:// \href{https://github.com/gaocegege}{\bf gaocegege} \\
% (380+ 关注者) \\
% LinkedIn://  \href{https://www.linkedin.com/in/gaocegege}{\bf gaocegege} \\

%%%%%%%%%%%%%%%%%%%%%%%%%%%%%%%%%%%%%%
%     COURSEWORK
%%%%%%%%%%%%%%%%%%%%%%%%%%%%%%%%%%%%%%

\section{修读课程}
\subsection{Graduate}
电路原理 \\
模拟电路设计 \\
数字电路分析与设计 \\
微机原理 \\
单片机原理及应用 \\
信号与系统 \\
verilog项目设计 \\
\sectionsep

%%%%%%%%%%%%%%%%%%%%%%%%%%%%%%%%%%%%%%
%     SKILLS
%%%%%%%%%%%%%%%%%%%%%%%%%%%%%%%%%%%%%%
\section{技能}
\sectionsep
\subsection{编程}
\location{超过 5000 行}
Python \textbullet{} C \\
\location{1000 - 5000 行}
C++ \textbullet{} MatLab \textbullet{} Shell \textbullet{} Verilog \\
\location{低于 1000 行}
Makefile \textbullet{} Lua \\ 
\sectionsep
% \section{技能}
% \sectionsep
% \subsection{编程}
% \location{超过 5000 行}
% Python \textbullet{} C \textbullet{} \\
% \location{1000 - 5000 行}
% C++ \textbullet{} MatLab \textbullet{} \LaTeX\ \\
% \location{低于 1000 行}
% HTML \textbullet{} Javascript \textbullet{} MatLab \textbullet{} Shell \textbullet{} Processing \\ 
% \sectionsep

\subsection{硬件协议}
\location{熟悉}
IIC \textbullet{} SPI \textbullet{} UART \\
\location{了解}
DMA \textbullet{} CAN \textbullet{} PCIE \\
\sectionsep

\subsection{软件}
\location{系统}
Freetos \textbullet{} Linux\\
\location{gui}
QT \textbullet{} PyQt5 \\
\location{机器学习}
\descript{框架:}Pytorch \\
\descript{算法:}CNN \textbullet{} MLP \textbullet{} SAC\\
\descript{开源项目:}Yolo \textbullet{} Ollama \\
\sectionsep

\subsection{技能证书}
\location{英语}
CET6 \\
\location{计算机}
全国计算机二级(C语言) \\
\sectionsep
% \subsection{云计算}
% \location{一般}
% Docker \textbullet{} Kubernetes \\
% \location{了解}
% Swarm \textbullet{} Moby \textbullet{} Linuxkit \textbullet{} HyperContainers \textbullet{} ClearContainers \textbullet{} Xen \textbullet{} KVM \textbullet{} Unikernel \\
% \sectionsep


%%%%%%%%%%%%%%%%%%%%%%%%%%%%%%%%%%%%%%
%
%     COLUMN TWO
%
%%%%%%%%%%%%%%%%%%%%%%%%%%%%%%%%%%%%%%

\end{minipage} 
\hfill
\begin{minipage}[t]{0.73\textwidth} 

%%%%%%%%%%%%%%%%%%%%%%%%%%%%%%%%%%%%%%
%     EXPERIENCE
%%%%%%%%%%%%%%%%%%%%%%%%%%%%%%%%%%%%%%

\section{实习经历}
\sectionsep
\runsubsection{南京丹迪克科技开发有限公司}
\descript{学生参与者,参与\href{https://github.com/purplefrigid/Zynq-Simple-Oscilloscope.git}{\bf 基于zynq7020的多功能示波器}}
\location{2023.01 - 2023.05 | 远程}
\vspace{\topsep}
\begin{tightemize}
    \item 调用锁相环ip,rom ip核实现dds信号发生器,调试adc采集功能,hdmi功能
    \item 在ps端在核1上移植freetos操作系统
    \item 在ps端核2上通过petalinux移植linux操作系统
    \item 在freetos上编写示波器核心功能,实现8路信号共同与单独显示,\\信号暂停,信号保存等功能
\end{tightemize}
\sectionsep

% \runsubsection{摩根士丹利}
% \descript{CIP 项目实习生}
% \location{2017.02-2017.08 | 上海}
% \begin{tightemize}
% \item 优化开源容器调度管理框架 treadmill 的调度器
% \item 实现与 Kubernetes 类似的调度模型,同时保留自身的树形结构
% \end{tightemize}
% \sectionsep

% \runsubsection{上海触宝信息技术有限公司}
% \descript{数据工程师(实习)}
% \location{2015.09-2015.09 | 上海}
% \begin{tightemize}
% \item 移植爬虫代码到新的平台,优化重写部分过期的爬虫
% \end{tightemize}
% \sectionsep

% \runsubsection{蚂蚁金服(杭州)网络技术有限公司}
% \descript{Java 研发工程师(实习)}
% \location{2015.07-2015.09 | 杭州}
% \begin{tightemize}
% \item 在支付宝国际事业团队从事海外直购业务开发
% \item 实现部分包裹清关的逻辑和后台管理的逻辑
% \end{tightemize}
% \sectionsep

%%%%%%%%%%%%%%%%%%%%%%%%%%%%%%%%%%%%%%
%     RESEARCH
%%%%%%%%%%%%%%%%%%%%%%%%%%%%%%%%%%%%%%
\section{项目}
\sectionsep
\runsubsection{\href{https://github.com/purplefrigid/Intelligent-chatbot-design-based-on-sg2002.git}{\bf 基于SG2002的智能聊天机器人}}
\descript{Maintainer}
\location{2025.03}
\begin{tightemize}
    \item 部署yolov5和ollama到SG2002上
    \item 部署kaldi到树莓派4b上实现语音识别
    \item 在SG2002小核上部署freetos,通过iic控制oled
    \item 通过mailbox实现SG2002大核和小核通信,通过ollama生成词实现不同表情切换
    \item 麦克风采集的声音由树莓派转换成文字通过uart发送给SG2002,
    \\SG2002通过ollama生成句子,由SYN6988语音合成芯片播放,并控制小核显示表情
    \end{tightemize}
\sectionsep

\runsubsection{\bf 基于rk3568和树莓派的智能监控系统设计}
\descript{Maintainer}
\location{2024.10}
\begin{tightemize}
    \item 在rk3568上移植uboot,Linux内核,buildroot根文件系统
    \item 添加设备树,编写摄像头驱动
    \item 移植ffmpeg,搭建流服务器,实现rtmp推流
    \item 在树莓派上移植yolov5
    \item 利用qt编写ui界面,显示经yolov5预测的监控视频
    \end{tightemize}
\sectionsep

\runsubsection{\href{https://github.com/purplefrigid/smart_car.git}{\bf 基于英飞凌TC264的竞速智能车设计}}
\descript{Owner}
\location{2021.7}
\begin{tightemize}
    \item 编写摄像头驱动,舵机驱动
    \item 智能车电机pid调参
    \item 摄像头采集图像处理:最大阈值法降噪,赛道中线寻找,赛道补线
    \item 智能出入库算法实现
    \end{tightemize}
\sectionsep

\runsubsection{\bf 基于stm32的智能送药小车设计}
\descript{Maintainer}
\location{2022.10}
\begin{tightemize}
    \item 编写舵机驱动,编码器电机驱动
    \item 基于Openmv的巡线算法设计
    \end{tightemize}
\sectionsep
% \section{项目与论文}
% \sectionsep
% \runsubsection{\href{https://github.com/caicloud/cyclone}{\bf Cyclone}}
% \descript{Maintainer}
% \location{2016.11}
% \begin{tightemize}
%     \item 基于 Docker 的持续集成与持续部署系统
%     \item 本科毕业设计,与才云科技合作开发,在 GitHub 上获得 \textbf{440 stars}
%     \item 调研其他开源实现,确定工作流程和架构选型,实现 YAML parser 和 Docker 的运行时集成
%     \end{tightemize}
% \sectionsep

% \runsubsection{\href{https://github.com/gaocegege/scrala}{\bf Scrala}}
% \descript{Owner}
% \location{2015.12}
% \begin{tightemize}
%     \item 使用 scala 实现的爬虫框架,灵感来自 scrapy
%     \item 在 GitHub 上获得 \textbf{70 stars}
%     \item 底层使用 Actor 模型取代 Python 中的异步模型
%     \end{tightemize}
% \sectionsep

%%%%%%%%%%%%%%%%%%%%%%%%%%%%%%%%%%%%%%
%     OPEN SOURCE
%%%%%%%%%%%%%%%%%%%%%%%%%%%%%%%%%%%%%%
% \section{专利与论文}
% \begin{tabular}{ll}
% \href{https://github.com/moby/moby/commits?author=gaocegege}{\bf moby/moby} & 实现 docker service ps -q 参数,与 swarmkit 更好集成 \\
% \href{https://github.com/opencontainers/runc/commits?author=gaocegege}{\bf opencontainers/runc} & 为了修复 \href{https://github.com/moby/moby/issues/27484}{moby/moby\#27484} 对上游进行的修改 \\
% \href{https://github.com/pingcap/tidb/commits?author=gaocegege}{\bf pingcap/tidb} & 在 travis 里引入了覆盖率测试; 实现 truncate 函数 \\
% \href{https://github.com/coala/coala-vs-code/commits/master?author=gaocegege}{\bf coala/coala-vs-code} & Visual Studio Code 上的插件,项目 maintainer \\
% \href{https://github.com/weijianwen/SJTUThesis/commits?author=gaocegege}{\bf weijianwen/SJTUThesis} & 为学士论文模板添加英文大摘要; 替换版权字体 \\
% \end{tabular}
% \sectionsep
% \section{开源贡献}
% \begin{tabular}{ll}
% \href{https://github.com/moby/moby/commits?author=gaocegege}{\bf moby/moby} & 实现 docker service ps -q 参数,与 swarmkit 更好集成 \\
% \href{https://github.com/opencontainers/runc/commits?author=gaocegege}{\bf opencontainers/runc} & 为了修复 \href{https://github.com/moby/moby/issues/27484}{moby/moby\#27484} 对上游进行的修改 \\
% \href{https://github.com/pingcap/tidb/commits?author=gaocegege}{\bf pingcap/tidb} & 在 travis 里引入了覆盖率测试; 实现 truncate 函数 \\
% \href{https://github.com/coala/coala-vs-code/commits/master?author=gaocegege}{\bf coala/coala-vs-code} & Visual Studio Code 上的插件,项目 maintainer \\
% \href{https://github.com/weijianwen/SJTUThesis/commits?author=gaocegege}{\bf weijianwen/SJTUThesis} & 为学士论文模板添加英文大摘要; 替换版权字体 \\
% \end{tabular}
% \sectionsep

%%%%%%%%%%%%%%%%%%%%%%%%%%%%%%%%%%%%%%
%     AWARDS
%%%%%%%%%%%%%%%%%%%%%%%%%%%%%%%%%%%%%%

\section{所获奖项} 
\begin{tabular}{rll}
2021         & 二等奖  & 全国大学生智能车大赛 \\
2021	     & 二等奖  & 大学生电子设计大赛 \\
2021	     & 奖学金  & 校科技创新奖学金 \\
2022	     & 三等奖  & 蓝桥杯嵌入式赛道 \\
2023         & 奖学金  & 校学业优秀奖学金\\
2024	     & 奖学金  & 校研究生奖学金 \\
\end{tabular}
\sectionsep

%%%%%%%%%%%%%%%%%%%%%%%%%%%%%%%%%%%%%%
%     PUBLICATIONS
%%%%%%%%%%%%%%%%%%%%%%%%%%%%%%%%%%%%%%

% \section{Publications} 
% \renewcommand\refname{\vskip -1.5cm} % Couldn't get this working from the .cls file
% \bibliographystyle{abbrv}
% \bibliography{publications}
% \nocite{*}

\end{minipage} 
\end{document}  \documentclass[]{article}
